\documentclass[11pt]{article}

\usepackage{amssymb}
\usepackage{amsmath}
\usepackage{fullpage}
\usepackage{graphicx}
\usepackage{multicol}

\newcommand{\mb}{\mathbf}

\title{\bf Control of Underactuated Vehicular Traffic}
\author{Stanislav Nikolov}

\begin{document}
\maketitle

\begin{multicols}{2}

% Misc stuff to talk about
% Traffic can actually smooth out by itself depending on the parameter governing the dynamics. Can show this analytically for linear dynamics, and demonstrate it for nonlinear dynamics.

% describe / motivate your system
\section{Introduction}
A recent study by the U.S. Treasury Department found that traffic in the United States wasted 1.9 billion gallons of gas. Other studies have suggested that peoplein the United States spent as many as 5 billion hours in traffic, resulting in as much as \$100 billion in lost productivity.  spend and 5 billion hours of productivity in the past year.

Traffic jams form for a variety of reasons. Disturbances, such as traffic accidents, or road bottlenecks cause vehicles to slow down, causing the vehicles behind them to slow down and eventually leading to a wave of traffic traveling upstream of the disturbance. However, traffic jams are often observed to form spontaneously without any apparent disturbance. Even if a platoon of vehicles starts out equally spaced with all cars moving at the same velocity, it is impossible for all vehicles to maintain the exact same speed. In an ideal world, the vehicles can maintain this configuration indefinitely, but in the real world, vehicles may stray from their desired velocity, causing a disturbance in the uniform velocities and spacings. Under certain conditions, this disturbance becomes amplified and leads to a self-sustaining traffic jam. % What kinds of conditions?

In order to talk about stabilizing vehicular traffic, we have to first present an idealized model of traffic flow and then discuss the relevant notions of stability for the model. We model traffic as $n$ vehicles driving in a single lane, with vehicle 1 at the lead, vehicle 2 directly after, and so on. Following \cite{Peng}, we define the following relevant variables:

\begin{itemize}
\item $x_i$ - the position of vehicle $i$.
\item $v_i$ - the velocity of vehicle $i$.
\item $a_i$ - the acceleration of vehicle $i$.
\item $\mb x_i$ - the state of vehicle $i$.
\item $u_i$ - the control input to vehicle $i$.
\item $R_i$ - the distance $x_{i-1} - x_i$ in front of vehicle $i$.
\end{itemize}

As in \cite{Peng}, we conside vehicles with idealized dynamics as well as those that have servo-loop dynamics. In idealized dynamics, the acceleration $a_i$ instantaneously takes on the value of the control input $u_i$, giving dynamics
\[\dot{\mb x}_i = \left[\begin{array}{c} \dot{x}_i\\ \dot{v}_i \end{array} \right] = \left[\begin{array}{c} v_i\\ u_i \end{array} \right] =  \left[\begin{array}{cc} 0 & 1\\ 0 & 0 \end{array} \right] \left[\begin{array}{c} x_i\\ v_i \end{array} \right] +  \left[\begin{array}{c} 0\\ 1 \end{array} \right]u_i. \]

In this paper, we will focus on {\em string stability}.

% describe the algorithmic approach you selected and why
\section{Algorithmic Approach}

LQR cost doesn't really handle rewarding high velocity. SGD can do any objective function.

\section{Results}

% list any surprises / difficulties that you encountered (so that we can all learn from them)
\section{Discussion}  

The qualitative behavior of traffic models is difficult to understand. Some models seem to form shocks and rarefactions and others seem to not. It is difficult to tell whether this is due to the parameter setting or an inherent limitation in the model.

Aw Rascle model seems more aggressive. It is also nonlinear. Hence shocks?

Numerical Instability (preventing collisios but not updating velocities and accelerations to 0). Leads to NaN scores for sgd.

Need small deltas for sgd for small ks.

Need more clever penalty function, since cars can be | o o o o o o o o o           o o o | (good, but high penalty because of gap) or | o ooo o o ooo o oo o o o ooo oo  oo| (bad but good penalty becase of no gap)

How to handle infinite cost for SGD. For example, if we have a 1/avg_velocity term in the cost function and all the cars stop, the cost will be infinite. The cars will all stop in finite time if the active gains are set to zero.

\end{multicols}

\end{document}
